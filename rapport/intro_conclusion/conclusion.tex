\section*{Conclusion}
\addcontentsline{toc}{section}{Conclusion}


\paragraph{}
La démarche employée est itérative. Dans l'approche MBSE, on divise le système en fonctions et sous fonctions, pour obtenir une structure d'arbre enraciné, que l'on confrontent d'un côté aux exigences et de l'autre aux composants qui assurent/perform ces fonctions. Le tracé de diagrammes dynamiques comme les EFFBD aident à visualiser le système en action, à décrire des scenarii, et donc à découvrir des fonctions qui auraient pu être oubliées et à régler le niveau de granuralité optimal au même titre que le reporting permis par CORE qui confronte functions, components, item et link dans une vue tabulaire efficace lors de la phase de détection des manques.  

\paragraph{}
Le reporting de CORE est aussi une aide précieuse pour la validation du projet en amont par le client et l'évaluation financière et ressources nécessaires à ce dernier. CORE est donc aussi un outil de communication convainquant.

\paragraph{}
Par manque de temps, les aspects "link" et "requirement" n'ont pu être traités pour aller au delà des objectifs de l'énoncé. Il manque probablement de plus à ce projet encore un certain nombre d'itérations. 

\paragraph{}
L'aspect Maintien en Conditions Opérationnelles(MCO) du système ATC n'a pas été abordé, celui du développement de ces systèmes non plus. En France, le MCO du système est assuré par les maintenances de la Direction des Opérations (DO) de la DGAC. Quant au développement, il relève de la Direction Technique et de l'Innovation(DTI) qui assure la MOA vis à vis des prestataires externes et le lien avec les partenaires, notamment dans le cadre de sa participation à certains "Work Package" du programme SESAR. En outre, comme indiqué dans l'introduction, nous avons choisi de faire abstraction des environnements "safety/security" et "phénomènes météo".   

\paragraph{}
Enfin, un dernier mot sur l'outil utilisé. CORE est un outil efficace pour aider à construire "the right system and the system right" mais trop lié au monde Windows. En particulier, une meilleure connaissance de l'outil auraient permis un gain de temps important dans le cadre du projet ROBAFIS bien que la version "university" n'autorise pas le travail collaboratif. CORE permet, en effet, d'assurer l'exhaustivité de la prise en compte des exigences par les fonctions et la création automatique des tableaux pour alimenter le rapport de développement Encore faut-il s'assurer que les exigences traduisent correctement les besoins.