\section*{Conclusion}
\addcontentsline{toc}{section}{Conclusion}


\paragraph{}
La démarche employée est itérative. Dans l'approche MBSE, on divise le système en fonctions et sous fonctions que l'on confrontent d'un côté aux exigences si ces dernières ont été saisie dans CORE et de l'autre aux composants qui assurent ces fonctions. Le tracé de diagrammes dynamiques comme les EFFBD aident à visualiser le système en action et donc à découvrir des fonctions qu'on auraient pu oublier et à régler le niveau de granuralité optimal au même titre que le reporting permis par CORE qui confronte fonctions, composant, item, link.  

\paragraph{}
Le reporting de CORE est aussi une aide précieuse pour la validation du projet en amont par le client et l'évaluation financière et ressources nécessaires à ce dernier.

\paragraph{}
Par manque de temps, les aspects link et requirement n'ont pu être traités. Mais, ils n'étaient pas explicitement demandés. Il manque de plus à ce projet encore de nombreuses itérations ; la précipitation est génératrice de bugs. Enfin, l'aspect maintien en conditions opérationnelles du système ATC n'a pas été abordé, celui du développement de ces systèmes non plus.

\paragraph{}
CORE semble donc aider à construire "the right system and the system right". En particulier permet d'assurer
l'exhaustivité de la prise en compte des exigences par les fonctions. Encore faut-il que les exigences traduisent correctement les besoins.