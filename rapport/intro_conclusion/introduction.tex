
\pagebreak

\section*{Introduction}
\addcontentsline{toc}{section}{Introduction}

L'exercice proposé a pour finalité la modélisation et l'analyse d'un système ATC simplifié dans un cadre MBSE. Cette approche permet, en effet, des vérifications précoces intervenant lors de la phase décroissante du cycle en V, juste avant l'implémentation.  

Les composants internes, constitutifs du système ATC considéré, comptent des acteurs humains, les contrôleurs, et des acteurs systèmes hardware et software. On citera les radars primaires, secondaires, les logiciels de traitement, de transport et d'affichage des données radar et plan de vol sans oublier les filets de sauvegarde. Le NMOC européen, Network Management Operation Centre, et le système de traitement de plan de vol national sont inclus dans le système étudié car ils participent à la fourniture du service de contrôle. Par contre, les composants humains, matériels et logiciel associés à l'aspect technique assurant notamment la maintenance des composants ne sont pas pris en compte dans le cadre de cet exercice.

Quant aux acteurs extérieurs, ce sont en particulier les pilotes, les compagnies aériennes, la référence horaire GPS, les aéronefs équipés de transpondeurs, le service météo. On pourrait ajouter les acteurs humains impactant la sûreté et la sécurité du système directement par le hacking du réseau ATC ou par brouillage des communications radio, par détournement de vols, etc. Ceci impacte sur les contraintes que subit le système sans oublier l'acteur environnemental des phénomènes météo. On choisit cependant de les ignorer également pour se focaliser sur l'objectif pédagogique principal de cet exercice. 







